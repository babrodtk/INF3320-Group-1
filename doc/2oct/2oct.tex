\documentclass[11pt]{beamer}
\usepackage[utf8]{inputenc}
\usepackage{amsmath}
\usepackage{xcolor,colortbl}
\usepackage{listings}
\usepackage{graphicx}
\usepackage{multirow}

\title{INF3320 Group Meeting}
\author{Jens Kristoffer Reitan Markussen}
\date{\today}
\subject{Informatics}
\begin{document}
	\begin{frame}
	    \frametitle{Obligatory Exercises}
	    \framesubtitle{Evaluation Point System for Source Code}
	    \begin{center}
	    \scalebox{0.55}{
		\begin{tabular}{ |l|l|l| }
			\hline
			Area & Points & Explanation \\ \hline
			\multirow{4}{*}{Compilers} 
			& \cellcolor{red!85} 0 & \cellcolor{red!85} Does not compile because of errors \\
 			& \cellcolor{red!85} 1 & \cellcolor{red!85} Does not compile because of linking problems
 			 solvable by student \\
 			& \cellcolor{green!85} 2 & \cellcolor{green!85} Compiles with warnings OR does not compile 
 			due to linking problems not solvable by student \\
	 		& \cellcolor{green!85} 3 & \cellcolor{green!85} Compiles without problems or
	 		 warnings\\ \hline
			\multirow{4}{*}{Runs}
 			& \cellcolor{red!85} 0 & \cellcolor{red!85} Runs not at all or does not do the task \\
 			& \cellcolor{red!85} 1 & \cellcolor{red!85} Starts, but breaks after some time or after 
 			some specific action due to major error \\
 			& \cellcolor{green!85} 2 &\cellcolor{green!85} Starts, but breaks after some time or after 
 			some specific action due to minor error \\
	 		& \cellcolor{green!85} 3 & \cellcolor{green!85} Runs without problem \\ \hline
			\multirow{5}{*}{Exercise}
 			& \cellcolor{red!85} 0 & \cellcolor{red!85} Exercise not solved or understood at all \\
 			& \cellcolor{red!85} 1 & \cellcolor{red!85} Good effort but did not get very far \\
 			& \cellcolor{red!85} 2 & \cellcolor{red!85} Main parts of exercise solved, but details 
 			are missing OR/AND insight seems to be missing \\
	 		& \cellcolor{green!85} 3 & \cellcolor{green!85} Exercise solved with minor errors/problems \\
	 		& \cellcolor{green!85} 4 & \cellcolor{green!85} Exercise solved \\ \hline
	 		\multirow{4}{*}{Code Quality}
 			& \cellcolor{red!85} 0 & \cellcolor{red!85} "Spaghetti code" - unreadable \\
 			& \cellcolor{red!85} 1 & \cellcolor{red!85} Poor: badly structured, no error/failure handling \\
 			& \cellcolor{green!85} 2 & \cellcolor{green!85} Acceptable: readable, sufficiently structured, 
 			most/main errors handled \\
	 		& \cellcolor{green!85} 3 & \cellcolor{green!85} Good: structured, errors handled, efficient, 
	 		easy readable \\ \hline
	 		\multirow{4}{*}{Comments}
 			& \cellcolor{red!85} 0 & \cellcolor{red!85} None \\
 			& \cellcolor{red!85} 1 & \cellcolor{red!85} Repeats code \\
 			& \cellcolor{green!85} 2 & \cellcolor{green!85} Bare minimum \\
	 		& \cellcolor{green!85} 3 & \cellcolor{green!85} Good \\
			\hline
		\end{tabular}
		}
	    \end{center}
    
  	\end{frame}

	\begin{frame}
	\frametitle{Source Control}
	\framesubtitle{GitHub}
	
	\begin{center}
	
	Easy to follow tutorial for setting up git on your machine: \\
	https://help.github.com/articles/set-up-git
	
	\end{center}
	Download a git client of choice if you don't like to use terminal commands
	\end{frame}	  	
  	
	\begin{frame}
	\frametitle{Source Control cont.}
	\framesubtitle{GitHub}
	Some useful commands
	\begin{itemize}
	\item clone - clones a remote repository from url
	\item pull - pull updates from remote
	\item push - push latest committed changes
	\item commit - commit changes made locally
	\item rm - remove a file from source control
	\item add - add a new file to source control
	\item status - prints out the current status of local repo.
	\end{itemize}
	
	Syntax:\\
	git [command] [options] \\
	Example:\\
	git clone git@github.com:jkrmc12/INF3320-Group-1.git \\
	cd INF3320-Group-1 \\
	git add doc/2oct/2oct.tex \\
	git commit \\
	git push \\
	
	\end{frame}  	
  	
	\begin{frame}
		\frametitle{Source Control cont.}
		\framesubtitle{GitHub}
		When using source control when working on the obligatory assignments you are required to use private repositories. \\ 
		On GitHub only public repositories are free, however you can upgrade to a students account and get five private repositories for free.\\
		https://github.com/edu
		
	\end{frame}	  	
  	
  	\begin{frame}
    		\frametitle{Exercise 5.1}
		\begin{center}
    		Describe the difference between flat, Gouraud, and Phong-shading.
		\end{center}		    		
  	\end{frame}

	\begin{frame}
    		\frametitle{Exercise 5.1}
    		\framesubtitle{Solution}
		\textbf{Solution:}	
		The main difference between shading models is the level at which lighting
calculations are done (primitive, vertex or fragment).
Flat shading evaluates the lighting model once for each triangle, and fills the triangle with
that colour.
Gouraud shading evaluates the lighting model at each of the three vertices and interpolates
the colour to determine the fragment colours inside the triangle.
Phong shading evaluates the lighting model per fragment. The normal vector is interpolated
(and normalized) over the triangle.		    		
		    
  	\end{frame}
  	
  	\begin{frame}
  		\frametitle{Exercise 5.2}
    		Given a point $\vec{p}$ on a surface with surface normal $\vec{n}$ and a light ray coming from a medium with refraction index $\eta _2$ and is refracted into a medium with refraction index $\eta _2$. The
incoming light ray makes the angle $\theta$ with the surface normal.
Make a small drawing and set up the relevant expressions for the refracted direction.
When do total reflection happen?
  	\end{frame}

	\begin{frame}
		\frametitle{Exercise 5.2}
		\framesubtitle{Solution}
		$\vec{r}$: Refracted direction \\
		$\vec{i}$: Incoming light \\
		$\vec{n}$: Surface normal \\
				
		Define $\vec{r}$ as:
		\begin{align}
		\vec{r} = \alpha \vec{i} + \beta \vec{n}
		\end{align}
		Since be know $\vec{r}$ lies in the plane spanned by the incoming light vector $\vec{i}$ and surface $\vec{n}$. \\
		Snell's Law
		\begin{align}
		\eta_1 \sin \theta = \eta_2 \sin \theta_{r}
		\end{align}
		$\theta_{r}$ is an angle the refracted direction vector $\vec{r}$ makes with $-\vec{n}$		
		
	\end{frame}	
	
	\begin{frame}
		\frametitle{Exercise 5.2}
		\framesubtitle{Solution cont.}
		
		Total reflection happens when two conditions are fulfilled:
		
		\begin{enumerate}
\item[(a)] The index of refraction of the medium the light is traveling from is greater than the
index of refraction of the medium the light is traveling to
$\eta_1 > \eta_2$
\item[(b)] The angle of incidence (angle between the light vector and surface normal) is grater
than the critical angle given by'
$\theta_c = \sin^{-1} \left(\frac{\eta_2}{\eta_1}\right)$
\end{enumerate}
	\end{frame}		
\end{document}
