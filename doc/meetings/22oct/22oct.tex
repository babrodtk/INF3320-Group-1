\documentclass[11pt]{beamer}
\usepackage[utf8]{inputenc}
\usepackage{amsmath}
\usepackage{amsfonts}
\usepackage{xcolor,colortbl}
\usepackage{listings}
\usepackage{graphicx}
\usepackage{multirow}
\usepackage{hyperref}
\hypersetup{
    bookmarks=true,         % show bookmarks bar?
    unicode=false,          % non-Latin characters in Acrobat’s bookmarks
    pdftoolbar=true,        % show Acrobat’s toolbar?
    pdfmenubar=true,        % show Acrobat’s menu?
    pdffitwindow=false,     % window fit to page when opened
    pdfstartview={FitH},    % fits the width of the page to the window
    pdftitle={My title},    % title
    pdfauthor={Author},     % author
    pdfsubject={Subject},   % subject of the document
    pdfcreator={Creator},   % creator of the document
    pdfproducer={Producer}, % producer of the document
    pdfkeywords={keyword1} {key2} {key3}, % list of keywords
    pdfnewwindow=true,      % links in new window
    colorlinks=true,       % false: boxed links; true: colored links
    linkcolor=red,          % color of internal links (change box color with linkbordercolor)
    citecolor=green,        % color of links to bibliography
    filecolor=magenta,      % color of file links
    urlcolor=cyan           % color of external links
}

\title{INF3320 Group Meeting}
\author{Jens Kristoffer Reitan Markussen}
\date{\today}
\subject{Informatics}
\begin{document}

	\begin{frame}
	\frametitle{Exercises - Shaders}
	\framesubtitle{From exercises14.pdf}
	Along with this exercise you are provided with a program to test your shaders. The program takes vertex shader as a first and fragment shader as the second command line parameter.
Simple rotation mechanism is available using keys 'w', 's', 'a' and 'd'. Different models are
available using keys '1', '2', '3'. Filename - ex14-1\_shaders.cpp. \\ \textit{PS: I made some modifications to the handout used last year, so remember to use the one from this git repo, not the one from last years semester page.}

	\end{frame}
	
	\begin{frame}
	\frametitle{Lighting equations}
	\begin{enumerate}
			
	\item[-] Ambient light $\quad I_{amb} = M_{amb}L_{amb}$
	\item[-] Diffuse light $\quad I_{diff} = M_{diff}L_{diff}(\hat{n}\cdot \hat{l})$
	\item[-] Specular light $\quad I_{spec} = M_{spec}L_{spec}(\hat{n}\cdot \hat{h}), \hat{h}=\frac{l+v}{||l+v||}$
\end{enumerate}
	Built-in GLSL variables reference:\\
	\url{http://mew.cx/glsl_quickref.pdf}
	
	\end{frame}		
	
	\begin{frame}
	\frametitle{Exercises - Shaders}
	\framesubtitle{From exercises14.pdf cont.}
	\begin{enumerate}
	\item Write a vertex and fragment shader that will assign a $(1,0,0)$ color to each fragment.
	\item Write a vertex and fragment shader that will replicate a GL\_FLAT behavior of the fixed pipeline.
	\item Write a vertex and fragment shader that will replicate a GL\_SMOOTH behavior of the fixed pipeline - i.e. Gouard shading.	
	\item Write a vertex and fragment shader that will perform Phong shading.
	\end{enumerate}

	\end{frame}

	\begin{frame}
		\frametitle{Exercises - Shaders}
	\framesubtitle{Extra, Hard}
	Use what you learned about refractions/reflections, and try to implement Fresnel shading.
	\begin{enumerate}
	\item Compute reflection and refraction vector using snell's law (GLSL already has functions for this, you just need to provide a $\eta$ value)
	\item Use the vectors as texture coordinates to fetch a reflected and refracted color value from a cube map (environment mapping)\\
	vec3 refl\_uv = reflect(-v, n);\\
	vec3 refr\_uv = refract(-v, n, eta);\\
	vec4 refl = textureCube(cube\_tex, -refl\_uv);\\
	vec4 refr = textureCube(cube\_tex, -refr\_uv);\\
	\item Compute ratio of reflectance using Schlick's approximation
	\begin{align*}
	R_F({\theta_i}) \approx R_F({\theta^0})+(1-R_F({\theta^0}))(1-\cos_{\theta_i})^5 
	\end{align*}
	\begin{align*}
	 R_F({\theta^0}) = \left(\frac{\eta_2-\eta_1}{\eta_2+\eta_1}\right)^2, \quad \cos_{\theta_i} = (v\cdot n)
	\end{align*}
	\end{enumerate}
	\end{frame}

\end{document}